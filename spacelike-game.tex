\documentclass{article}
\usepackage{amsmath}
\usepackage{amssymb}
\usepackage{hyperref}
\begin{document}

\title{Relativistic Spacetime and Non-Causal Game Theory}
\author{}
\date{}
\maketitle

\section{Introduction}
This document explores the incorporation of relativistic spacetime intervals into game theory, with a focus on scenarios where causality is challenged, leading to potential reverse influence of outcomes on decision-making processes.

\section{Spacetime Geometry and Relativity Basics}
In special relativity, the position and time of an event are represented by a four-vector:
\begin{equation}
    x^\mu = (ct, x, y, z),
\end{equation}
where $c$ is the speed of light. The spacetime interval $s^2$ is defined as:
\begin{equation}
    s^2 = (ct)^2 - x^2 - y^2 - z^2.
\end{equation}
Depending on the sign of $s^2$, the interval is classified as:
\begin{itemize}
    \item $s^2 > 0$: Time-like interval (causal relationship possible).
    \item $s^2 = 0$: Light-like interval (events lie on the light cone).
    \item $s^2 < 0$: Space-like interval (no causal relationship).
\end{itemize}
In space-like intervals, events may appear simultaneous or even reversed in causality in certain reference frames.

\section{Game Theory in Non-Causal Settings}
Consider a simple two-player game with players $A$ and $B$, where:
\begin{itemize}
    \item Player $A$ makes a decision $a$ at event $E_A$.
    \item Player $B$ makes a decision $b$ at event $E_B$.
\end{itemize}
The events $E_A$ and $E_B$ are separated by a space-like interval, implying no clear causal order between them.

The payoff functions for the players are:
\begin{equation}
    U_A(a, b), \quad U_B(a, b),
\end{equation}
representing their respective utilities based on actions $a$ and $b$. In a non-causal setting, decisions may depend on outcomes that are, in principle, influenced retrocausally.

\subsection{Information Flow and Strategy Dependence}
Under relativistic constraints, we assume:
\begin{enumerate}
    \item \textbf{Action dependence on outcomes}: Player $A$'s decision $a$ depends not only on local information $I_A$ but also on $b$, which is determined at $E_B$.
    \item \textbf{Strategy representation}: Let $a = f_A(I_A, b)$ and $b = f_B(I_B, a)$.
    \item \textbf{Reverse causality}: Player $B$'s decision $b$ can also be influenced by $a$, forming a closed-loop dependence.
\end{enumerate}

\section{Fixed-Point Analysis}
In the non-causal framework, equilibrium strategies can be analyzed through fixed-point equations. Suppose:
\begin{equation}
    a = f_A(b), \quad b = f_B(a),
\end{equation}
where $f_A$ and $f_B$ are strategy functions. Solving for the fixed points gives:
\begin{equation}
    a^* = f_A(b^*), \quad b^* = f_B(a^*).
\end{equation}

\subsection{Example}
Let the payoff functions be:
\begin{equation}
    U_A(a, b) = -a^2 + 2ab, \quad U_B(a, b) = -b^2 + 2ab.
\end{equation}
Assume linear strategies:
\begin{equation}
    a = \alpha b, \quad b = \beta a.
\end{equation}
Substituting, we have:
\begin{equation}
    a = \alpha (\beta a) \implies a(1 - \alpha \beta) = 0.
\end{equation}
The solutions are:
\begin{equation}
    a = 0 \quad \text{or} \quad \alpha \beta = 1.
\end{equation}
The corresponding $b$ values are $b = \beta a$. This demonstrates the calculation of fixed points under non-causal influences.

\section{Interpretation and Physical Implications}
In a relativistic framework:
\begin{itemize}
    \item Fixed points represent equilibria where decisions and payoffs are balanced.
    \item Under space-like intervals, actions may mutually influence each other despite the absence of a causal order.
    \item This analysis parallels retrocausal effects observed in quantum mechanics, such as time-symmetric interpretations.
\end{itemize}

\section{Conclusion}
Integrating relativistic spacetime into game theory introduces intriguing possibilities for 
decision-making under non-causal conditions.
\end{document}
